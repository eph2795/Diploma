\documentclass[14pt, a4paper, oneside, bold]{extarticle}

\usepackage[english, russian]{babel}
\usepackage[utf8]{inputenc}
\usepackage{amsmath}
\usepackage{amssymb}
\usepackage{graphicx}
\usepackage{multicol}
\usepackage[14pt]{extsizes}
\usepackage{listingsutf8}
\usepackage{color}
\usepackage{wrapfig}
\usepackage{enumerate}
\usepackage{amsthm}
\usepackage{indentfirst}

\usepackage{setspace}
\singlespacing
\onehalfspacing
\doublespacing

\allowdisplaybreaks
\binoppenalty = 10000
\relpenalty = 10000
\textheight = 23cm
\textwidth = 20cm
\oddsidemargin = 0pt
%\topmargin = -1.5cm
\parskip = 0pt
\tolerance = 2000
\flushbottom

\usepackage[left=30mm, right=20mm, top=20mm, bottom=30mm, nohead, footskip=10mm]{geometry}
% bottom=2cm, bindingoffset=0cm]

\pagestyle{plain}

\renewcommand{\thesection}{\arabic{section}}

\begin{document}

\allowdisplaybreaks[1]

\begin{titlepage}
\setstretch{1}
\begin{center}
\ \vspace{-1.5cm}

\includegraphics[width=0.5\textwidth]{msu_logo.eps}\\
{\bfseries Московский государственный университет имени М.В. Ломоносова \\
Факультет вычислительной математики и кибернетики\\
Кафедра исследования операций}

\vspace{3cm}

{\Large Лаврухин Ефим Валерьевич}

\vspace{1cm}

{\Large\bfseries
Применение нейронных сетей для сегментации томографических изображений геологических пород\\}

\vspace{1cm}

{\textbf \large МАГИСТЕРСКАЯ ДИССЕРТАЦИЯ}
\end{center}

\vfill

\begin{flushright}
  \textbf{Научный руководитель:}\\
  к.ф.-м.н., доцент\\
  Д.В.~Денисов
\end{flushright}

\vfill

\begin{center}
Москва, 2018
\end{center}

\vspace{1cm}

\enlargethispage{2\baselineskip}

\end{titlepage}


\setstretch{1.5}

\newpage

\section*{Введение}

\textbf{Актуальность темы исследования.} В настоящее время добыча полезных ископаемых требует большое количество 
данных о разрабатываемых породах-коллекторах. Эти данные получают в том числе с помощью методов цифровой петрофизики, которые работают с  
2-D или 3-D изображениями, сделанными с помощью рентгеновской томографии \cite{1}. Большинство изображений строения пород представлены в градациях серого, которые указывают на интенсивность поглощения рентгеновских лучей. На практике любой метод численного расчета характеристик исходных пород состоит из нескольких отдельных этапов. 
И первый этап -- это сегментация входного изображения, разделение его
на несколько различных фаз по плотности вещества. В простейшем случае выполняется бинаризация -- разделение на твёрдую породу и поры \cite{2}. 

\textbf{Цель} данной работы -- применить методы глубинного обучения, а именно полносвёрточные нейронные сети, для задачи сегментации изображений геологических пород.

\textbf{Научная новизна.} В настоящее время существует большое количество методов сегментации. Они существенно отличаются в используемых предположениях о входных изображениях и математическом аппарате. Вот некоторые из них: градиентные \cite{16}, морфологические, случайные поля \cite{14}, \cite{15}, методы Монте-Карло \cite{13}. У этих методов есть ряд достоинств: присутствует математическая формализация, относительная простота постановки задачи, интерпретируемость результатов. Но в то же время все они обладают серьёзным недостатком -- в них присутствуют гиперпараметры, которые сильно влияют на качество результата. Это делает затруднительным их применение без оператора, который контролирует процесс и подбирает нужные значения параметров для  конкретных входных данных. 

Относительно недавно появились методы сегментации с использованием машинного обучения. Они так же применяются и в сегментации изображений геологических пород \cite{3}, \cite{4}, \cite{5}. Глубинное обучение - это подкласс моделей машинного обучения, в которых используется сложная многоуровневая композиция слоёв для извлечения нелинейных признаков исходного объекта. Главные преимущества этих моделей - это высокое качество(выше, чем у других методов, решающих аналогичные задачи) и полная автономность обучения(для того, чтобы построить качественную модель на имеющихся данных не требуется участие человека).

Сейчас свёрточные нейронные сети являются, фактически, state-of-the-art в задачах обрабоки изображений \cite{6} и используются во многих прикладных областях: биологии, медицине, распознавании образов \cite{11}, \cite{12}. На текущий момент выпущено достаточно много работ, в которых исследуются методы глубинного обучения для решения связанных с сегментацией пород задач. В частности для  построения стохастической реконструции пород с последующим моделированием физических свойств \cite{7}, \cite{8}, \cite{9}, \cite{10}. Но статей, в которых глубинное обучение применяется для сегментации геологических пород, крайне мало. 

В ходе работы для достижения поставленной цели решались следующие 
\textbf{задачи}:
\begin{enumerate}
	\item Выбор полносвёртночной архитектуры нейронной сети, 
	которая выполняет сегментацию изображений томограмм.
	\item Решение проблемы отсутствия размеченных обучающих данных.
	\item Построение стабильного алгоритма обучения сети.
	\item Учёт внутри модели 3-D структуры входных данных.
	\item Сравнение полученных результатов с результатами других 			моделей для сегментации геологических изображений.
	\item Сравнение физических характеристик, полученных с помощью 
	отсегментированных моделью изображений, с характеристиками 				исходного образца, вычисленными с помощью физических симуляций.
\end{enumerate}
  
\newpage

\section{Постановка задачи}

\section{Заключение}


\addcontentsline{toc}{chapter}{Литература}
\begin{thebibliography}{99}

\bibitem{1} S. Karimpoulia
	, P. Tahmasebib,
	, H. L. Ramandic
	, P. Mostaghimid
	, M. Saadatfar, 
	``Stochastic modeling of coal fracture network by direct use of microcomputed
tomography images'', International Journal of Coal Geology 179, 153-163, 2017.

\bibitem{2} Jeff T. Gostick, 
	``Versatile and efficient pore network extraction method using marker-based watershed segmentation'', Physical Review E 96, 2017.

\bibitem{3} S. Chauhan
	, W. Rühaak,
	, H. Anbergen
	, A. Kabdenov
	, M. Freise
	, T. Wille
	, I. Sass,
	``Phase segmentation of X-ray computer tomography rock images
using machine learning techniques: an accuracy
and performance study'', Solid Earth, 7, 1125–1139, 2016.
	
\bibitem{4} F. Khan
	, F. Enzmann,
	, M. Kersten, 
	``Multi-phase classification by a least-squares support vector machine
approach in tomography images of geological samples'', Solid Earth, 7, 481–492, 2016.

\bibitem{5} S. Chauhan
	, W. Rühaak,
	, F. Khan
	, F. Enzmann
	, P. Mielke
	, I. Sass,
	``Processing of rock core microtomography images: Using seven
different machine learning algorithms'', Computers \& Geosciences, 86, 120-128, 2016.

\bibitem{6} O. Ronneberger
	, P. Fischer
	, T. Brox,
	``U-Net: Convolutional Networks for Biomedical
Image Segmentation'', arXiv:1505.04597v1, 2015.

\bibitem{7} L. Mosser
	, O. Dubrule,
	, Martin J. Blunt, 
	``Reconstruction of three-dimensional porous media
using generative adversarial neural networks'', 
arXiv:1704.03225v1, 2017

\bibitem{8} Brian L. DeCost
	, T. Francis
	, Elizabeth A. Holm, 
	``Exploring the microstructure manifold: image
texture representations applied to ultrahigh 
carbon steel microstructures'', 
arXiv:1702.01117v2, 2017

\bibitem{9} Ruijin Cang
	, Yaopengxiao Xu
	, Shaohua Chen
	, Yongming Liu
	, Yang Jiao
	,M. Yi Ren,
	`Microstructure Representation and
Reconstruction of Heterogeneous Materials via
Deep Belief Network for Computational Material 
Design'', 
arXiv:1612.07401v3, 2017

\bibitem{10} N. Lubbers
	, T. Lookman
	, K. Barros,
	``Inferring low-dimensional microstructure representations using
convolutional neural networks'', 
arXiv:1611.02764v1, 2016

\bibitem{11} A. S. Razavian
	, H. A. Josephine
	, S. S. Carlsson,
	``CNN Features off-the-shelf: an Astounding Baseline for Recognition'', 
arXiv:1403.6382v3, 2014.

\bibitem{12} Kwang Moo Yi
	, Eduard Trulls
	, Vincent Lepetit
	, Pascal Fua,
	``LIFT: Learned Invariant Feature Transform'', 
arXiv:1603.09114v2, 2016.

\bibitem{13} С.А. Эль-Хатиб,
	``Сегментация изображений с мопощью смешаного и экспоненциального алгоритмов роя частиц'', 
Информатика и кибернетика, 1, 2015.

\bibitem{14} H. Deng
	, D.A. Clausi, 
	``Unsupervised image segmentation using a simple MRF model with a new implementation scheme'', 
Pattern Recognition, 37, 2323-2335, 2004.

\bibitem{15} H. Deng
	, D.A. Clausi, 
	``Image segmentation using Markov Random Field Model in Fully Parallel Cellular Nerwork Architecture'', 
Real-time Imaging, 6, 195-211, 2000.

\bibitem{16} H. Deng
	, D.A. Clausi, 
	``Improved Workflow for Unsupervised Multiphase Image Segmentation'', 
arXiv:1710.0967, 2017.

\end{thebibliography}

\end{document}
